%-------------------------------------------------------------------------------
%	SECTION TITLE
%-------------------------------------------------------------------------------
\cvsection{Projects}


%-------------------------------------------------------------------------------
%	CONTENT
%-------------------------------------------------------------------------------
\begin{cventries}

  %---------------------------------------------------------
  \cventry
  {Team Member} % Role
  {Penetration Testing Project} % Name of Project
  {Singapore} % Location
  {Jan 2021 - May 2021} % Date(s)
  {
    \begin{cvitems} % Description(s) of experience/contributions/knowledge
      \item {Conducted Penetration Testing for one of the IT syetems hosted in NUS, using both automated testing and manual testing}
      \item {Participated in scoping meetings with stakeholders involved}
      \item {Wrote a report and gave a presentation on the findings to stakeholders in NUS}
    \end{cvitems}
  }
  \cventry
  {Developer (Django, SQLite, Web Security)} % Role
  {Vulnerability Website \textmd{\em\tiny(\url{https://github.com/joelczk/vulnerability-website},)}} % Name of Project
  {Singapore} % Location
  {May 2020 - PRESENT} % Date(s)
  {
    \begin{cvitems} % Description(s) of experience/contributions/knowledge
      \item {Colloborated with Prof. Sufatrio from NUS to create a vulnerable website and hardened website to allow easy deployment for educational purposes}
      \item {Created a vulnerable version of the website with vulnerabilties such as SQL Injection attacks and CSRF vulnerabilities}
      \item {Created a hardened version of the website using in-built Django libraries to ensure \textbf{proper network headers for the website, as well as, automated code analysis and file scanning functionalities using Github Actions}}
    \end{cvitems}
  }
  \cventry
  {Project Lead (Python, Flask, PostresSQL)} % Role
  {LumiRandom \textmd{\em\tiny(\url{https://github.com/joelczk/LumiRandom})}} % Name of Project
  {Singapore} % Location
  {Jan 2019 - May 2019} % Date(s)
  {
    \begin{cvitems} % Description(s) of experience/contributions/knowledge
      \item {Worked on web project to simulate NUS's module management website as part of CS2102(Introduction to Database) project. using \textbf{Python as the backend language and PostgresSQl as the database of choice}}
      \item {Colloborated with a team of 5 to implement the database schema and the backend of the project}
      \item {Gave a short presentation of the database schema and the implemetation of the project to the tutors and received Distinction for this project}
    \end{cvitems}
  }
  %---------------------------------------------------------

  %---------------------------------------------------------
  % \cventry
  %    {Developer (Elixir, Phoenix Framework, TypeScript, React, Redux)} % Role
  %    {Source Academy \textmd{\em\tiny(\url{https://github.com/source-academy/})}} % Name of Project
  %    {Singapore} % Location
  %    {Jan 2018 - PRESENT} % Date(s)
  %    {
  %      \begin{cvitems} % Description(s) of experience/contributions/knowledge
  %      	\item {Ironing out bugs in the back-end of the website and interpreter engine of The Source (a subset of Javascript focussing on its functional programming paradigm), the language used in CS1101S, a module based on MIT 8.001 course and the famous SICP book.}
  %      	\item {Migrating the current version of the backend to Elixir 1.6 and Phoenix 1.3.}
  %      \end{cvitems}
  %    }
  %---------------------------------------------------------

  %\cventry
  %    {Builder and Developer (OpenWRT, Shell Script)} % Role
  %    {Wi-Fi Extender and Load-balancer \textmd{\em\tiny (\url{https://github.com/indocomsoft/Maker-Portfolio})}} % Name of Project
  %    {Singapore} % Location
  %    {Jan 2016 - Oct 2016} % Date(s)
  %    {
  %      \begin{cvitems} % Description(s) of experience/contributions/knowledge
  %      	\item {Built a system using Wi-Fi routers connected via powerline adaptors to extend coverage, but one Wi-Fi router is used as a load-balancer, utilising several other Wi-Fi routers to obtain higher speed on a Wi-Fi network with per-device speed limit.}
  %      \end{cvitems}
  %    }
  %---------------------------------------------------------
\end{cventries}
